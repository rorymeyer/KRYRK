% Options for packages loaded elsewhere
\PassOptionsToPackage{unicode}{hyperref}
\PassOptionsToPackage{hyphens}{url}
%
\documentclass[
]{book}
\usepackage{amsmath,amssymb}
\usepackage{lmodern}
\usepackage{iftex}
\ifPDFTeX
  \usepackage[T1]{fontenc}
  \usepackage[utf8]{inputenc}
  \usepackage{textcomp} % provide euro and other symbols
\else % if luatex or xetex
  \usepackage{unicode-math}
  \defaultfontfeatures{Scale=MatchLowercase}
  \defaultfontfeatures[\rmfamily]{Ligatures=TeX,Scale=1}
\fi
% Use upquote if available, for straight quotes in verbatim environments
\IfFileExists{upquote.sty}{\usepackage{upquote}}{}
\IfFileExists{microtype.sty}{% use microtype if available
  \usepackage[]{microtype}
  \UseMicrotypeSet[protrusion]{basicmath} % disable protrusion for tt fonts
}{}
\makeatletter
\@ifundefined{KOMAClassName}{% if non-KOMA class
  \IfFileExists{parskip.sty}{%
    \usepackage{parskip}
  }{% else
    \setlength{\parindent}{0pt}
    \setlength{\parskip}{6pt plus 2pt minus 1pt}}
}{% if KOMA class
  \KOMAoptions{parskip=half}}
\makeatother
\usepackage{xcolor}
\IfFileExists{xurl.sty}{\usepackage{xurl}}{} % add URL line breaks if available
\IfFileExists{bookmark.sty}{\usepackage{bookmark}}{\usepackage{hyperref}}
\hypersetup{
  pdftitle={How to Make the Most Out of Your Summer in Ann Arbor},
  pdfauthor={KRYRK},
  hidelinks,
  pdfcreator={LaTeX via pandoc}}
\urlstyle{same} % disable monospaced font for URLs
\usepackage{longtable,booktabs,array}
\usepackage{calc} % for calculating minipage widths
% Correct order of tables after \paragraph or \subparagraph
\usepackage{etoolbox}
\makeatletter
\patchcmd\longtable{\par}{\if@noskipsec\mbox{}\fi\par}{}{}
\makeatother
% Allow footnotes in longtable head/foot
\IfFileExists{footnotehyper.sty}{\usepackage{footnotehyper}}{\usepackage{footnote}}
\makesavenoteenv{longtable}
\usepackage{graphicx}
\makeatletter
\def\maxwidth{\ifdim\Gin@nat@width>\linewidth\linewidth\else\Gin@nat@width\fi}
\def\maxheight{\ifdim\Gin@nat@height>\textheight\textheight\else\Gin@nat@height\fi}
\makeatother
% Scale images if necessary, so that they will not overflow the page
% margins by default, and it is still possible to overwrite the defaults
% using explicit options in \includegraphics[width, height, ...]{}
\setkeys{Gin}{width=\maxwidth,height=\maxheight,keepaspectratio}
% Set default figure placement to htbp
\makeatletter
\def\fps@figure{htbp}
\makeatother
\setlength{\emergencystretch}{3em} % prevent overfull lines
\providecommand{\tightlist}{%
  \setlength{\itemsep}{0pt}\setlength{\parskip}{0pt}}
\setcounter{secnumdepth}{5}
\usepackage{booktabs}
\ifLuaTeX
  \usepackage{selnolig}  % disable illegal ligatures
\fi
\usepackage[]{natbib}
\bibliographystyle{plainnat}

\title{How to Make the Most Out of Your Summer in Ann Arbor}
\author{KRYRK}
\date{2022-07-21}

\begin{document}
\maketitle

{
\setcounter{tocdepth}{1}
\tableofcontents
}
\hypertarget{introduction-to-the-guide}{%
\chapter*{Introduction to the Guide}\label{introduction-to-the-guide}}
\addcontentsline{toc}{chapter}{Introduction to the Guide}

As the MBAn program starts at the end of June, we feel that it is important for students to have a comprehensive guide on how to make the most out of their summer in Ann Arbor as they kick-start their Michigan experience. Everyone is coming from a different place, whether it be from the same state, the same country, or a completely different country. The summer is a time to transition to the recruiting season, so our goal is to make sure that students feel settled, at home, and make friends who will be there for them for the program's length and hopefully beyond. While school is the primary focus of our time here, it is vital to students' mental health that they are able to enjoy their free time outside of class and easily navigate their environment. So, we want to give students an easy-access guide to the different facets of life in Ann Arbor.

In this project, we will create chapters that we feel are the main aspects of transitioning to Ann Arbor. We want to make it easier for future MBAn students to succeed in their program while having lots of fun. The first chapter is dedicated to introducing the authors of the project as well as introducing the purpose of the report. The remainder of the chapters include information about Move-In, Academics, Shopping, Food/Entertainment, and Commuting.

\hypertarget{about-us}{%
\chapter*{About Us}\label{about-us}}
\addcontentsline{toc}{chapter}{About Us}

{[} short paragraph introducing the team {]}

\hypertarget{kami}{%
\section{Kami}\label{kami}}

{[} Kami's about me page {]}

\hypertarget{raj}{%
\section{Raj}\label{raj}}

{[} Raj's about me page {]}

\hypertarget{yuqi}{%
\section{Yuqi}\label{yuqi}}

{[} Yuqi's about me page {]}

\hypertarget{rory}{%
\section{Rory}\label{rory}}

{[} Rory's about me page {]}

\hypertarget{kian}{%
\section{Kian}\label{kian}}

{[} Kian's about me page {]}

\hypertarget{table-of-contents}{%
\chapter*{Table of Contents}\label{table-of-contents}}
\addcontentsline{toc}{chapter}{Table of Contents}

\begin{enumerate}
\def\labelenumi{\arabic{enumi}.}
\tightlist
\item
  \protect\hyperlink{introduction}{Introduction to Ann Arbor}
\item
  \protect\hyperlink{move-in}{Moving In}
\item
  \protect\hyperlink{academics}{Academics}
\item
  \protect\hyperlink{shopping}{Shopping}
\item
  \protect\hyperlink{food-entertainment}{Food \& Entertainment}
\item
  \protect\hyperlink{commuting}{Commuting}
\end{enumerate}

\hypertarget{introduction-to-ann-arbor}{%
\chapter{Introduction to Ann Arbor}\label{introduction-to-ann-arbor}}

\hypertarget{welcome-to-ross}{%
\section{Welcome to Ross}\label{welcome-to-ross}}

\hypertarget{move-in}{%
\chapter{Move-In}\label{move-in}}

\hypertarget{academics}{%
\chapter{Academics}\label{academics}}

\hypertarget{shopping}{%
\chapter{Shopping}\label{shopping}}

\hypertarget{food-entertainment}{%
\chapter{Food \& Entertainment}\label{food-entertainment}}

\hypertarget{commuting}{%
\chapter{Commuting}\label{commuting}}

  \bibliography{book.bib,packages.bib}

\end{document}
